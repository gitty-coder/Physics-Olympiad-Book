\hspace{1cm} International Physics Olympiad (IPhO) since its commencement in 1967 has been the highest international platform for high school students from all over the world to test their knowledge and skills of theoretical and experimental Physics. Every year, through rigorous rounds of examinations, each country selects their top five contestants who will represent their country in the IPhO and compete with students from all the participating countries. Getting a chance to participate and represent one’s nation in a widely recognized event like IPhO is a lifelong achievement and glory by itself. Although much later, Nepal too put its foot into the arena for the first time in 2007 and has been regularly participating in the event every year. In these 14 years of participation history, Nepal has bagged 3 bronze medals and 7 honorable mentions. We are surrounded by India and China, the two big giants of IPhO who always make to the top five. Of course, we cannot compare to them, but looking at what we could achieve and what we have achieved, Nepal’s performance is not satisfactory. The simple and straightforward reason is that our preparation and training are not enough. And the sadder thing is that we have not made any tangible efforts in creating any kind resources that will systematically guide the potential and capacity of thousands of students across the country to compete in the Nepal Physics Olympiad (NePhO), a two round selection process conducted by Nepal Physical Society (NPS). We will have a better team only if we prepare in advance and in large number so that the best of the best will find place in the national team.     \\
 
\hspace{1cm} The purpose of this book, therefore, is to systematically guide +2 students and prepare them for the Nepal Physics Olympiad (NePhO). The content of this book and the types of problems are also organized as per the syllabus of the NePhO. Most of the problems are taken in their original form from a vast literature we came across while we were preparing for NePhO. Because of our research and thorough practice of the varieties of problems that show up in the Olympiads, we have shortlisted the most important ones here. Each chapter is divided into five categories. The key concepts and formulae section summarizes the important formulae and deep concepts related to that chapter. The Bridging problems are designed to reflect the overall gist of that chapter. In the level 1 problems w/ solution section, there are less difficult problems with stepwise guided solutions which are intended to motivate the beginners. Similarly, in the level 2 problems w/ solutions, there are more challenging problems with stepwise solutions which are intended to test the clarity and depth of understanding of a particular concept. Finally, in the Supplementary problems w/ hints section there are mixed problems with hints as necessary which will help students to practice further and excel the material of the chapter.\\

\hspace{1 cm} Care has been taken to eliminate typos, but we realise that errors might have crept in, despite the multiple proof readings and reviews of the original manuscript. We will feel grateful to the reader for their valuable suggestions and notice if they find any errors in the book.\\

\hspace{1cm} Finally, we acknowledge the …..         
